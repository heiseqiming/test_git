\PassOptionsToPackage{quiet}{fontspec}
\documentclass[a4paper]{book}
\usepackage[margin=2cm]{geometry}
\usepackage{inputenc}
\usepackage{ctex}
\usepackage{textcomp}%使用千分号需要导入的宏包
\usepackage{graphicx}

\usepackage{amsmath,amssymb,amsthm}


\usepackage{framed}
\usepackage{xcolor}
\colorlet{shadecolor}{gray!20}

\newcommand{\svgraybox}[1]{
\fboxsep=12pt\relax
\begin{shaded}
\vspace{6pt}
\begin{list}{}{
\setlength{\leftmargin}{12pt}
\setlength{\rightmargin}{12pt}
\setlength{\topsep}{6pt}
\relax
}
\item
#1
\end{list}
\vspace{6pt}
\end{shaded}
}

\begin{document}
\section{2.2 The area of a region between two graphs expressed as an integral}

If two functions f and g are related by the inequality (x) ≤ g(x) for all x in an interval [a, b]9 we write f < g on [a, b]. Figure 2.1 shows two examples. If $f<g$ on [a, b]9 the set S consisting of all points (x, y) satisfying the inequalities

f(x) ≤ y < g(×) > a<x<b,

is called the region between the graphs off and g. The following theorem tells us how to express the area of S as an integral.

Figure 2.1 The area of a region between two graphs expressed as an integral: a(S) ----- ∫* [g(x) -f(x)] dx.

Worked examples

theorem 2.1. Assume f and g are integrable and satisfy f ≤ g on [<a, b∖. Then the region S between their graphs is measurable and its area a(S) is given by the integral

(2.1)

Proof Assume first that $f$ and $g$ are nonnegative, as shown in Figure 2.1 (a). Let $F$ and $G$ denote the following sets:

F = {(*, y)∖a<x<b,Q<y <∕(x)},

G = {(x, y)∖a<x<b,0<y< g(x)}.

That is, G is the ordinate set of g, and F is the ordinate set off minus the graph of f The region S between the graphs off and g is the difference S = G — F. By Theorems 1.10 and 1.11, both F and G are measurable. Since F ⊂ G, the difference S = G — F is also measurable, and we have

a(S) = a(G) - a(F) = P g(x) dx - P∕(x) dx = P [g(x) -∕(x)] dx .

da	da	Ja

This proves (2.1) when f and g are nonnegative.

Now consider the general case where f ≤ g on [a, b], but f and g are not necessarily nonnegative. An example is shown in Figure 2.1(b). We can reduce this to the previous case by sliding the region upward until it lies above the x-axis. That is, we choose a positive number c large enough to ensure that 0 ≤ ∕(x) + c ≤ g(x) + c for all x in [a, b]. By what we have already proved, the new region T between the graphs of f + c and g + c is measurable, and its area is given by the integral

a(T) = £ [(g(x) + c) - (∕(x) + c)] dx = £ [g(x) - ∕(χ)] dx .

But T is congruent to S'; so S is also measurable and we have α(S) = a(T) = £ [g(x) - ∕(x)] dx .

da

This completes the proof.
\end{document}
